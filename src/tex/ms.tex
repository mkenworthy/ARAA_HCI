\documentclass[letterpaper]{ar-1col}
\usepackage{showyourwork}
\usepackage[letterpaper]{geometry}

\usepackage{natbib}
\usepackage{amsmath}
\usepackage{color}
\usepackage{hyperref}
\hypersetup{hidelinks}

\usepackage{graphbox}
\newcommand{\suz}[1]{\textcolor{magenta}{#1}}
\newcommand{\dan}[1]{\textcolor{green}{#1}}

\setcounter{secnumdepth}{4}
\usepackage{url}


\usepackage{lipsum}  


% Metadata Information
\jname{Annu. Rev. Astron. Astrophys.}
\jvol{AA}
\jyear{2024}
\doi{10.1146/TBD}

% autoref formatting
\def\sectionautorefname{Section}
\let\subsectionautorefname\sectionautorefname
\let\subsubsectionautorefname\sectionautorefname

% macros
\newcommand{\apjl}{Astrophysical Journal Letters}
\newcommand{\aj}{Astronomical Journal}
\newcommand{\apj}{Astrophysical Journal}
\newcommand{\apjs}{Astrophysical Journal Supplement}
\newcommand{\pasp}{Publications of the Astronomical Society of the Pacific}
\newcommand{\jgr}{Journal of Geophysical Research}
\newcommand{\aap}{Astronomy and Astrophysics}
\newcommand{\mnras}{Monthly Notices of the Royal Astronomical Society}
\newcommand{\actaa}{Acta Astronomica}
\newcommand{\nat}{Nature}
\newcommand{\prl}{Physical Review Letters}
\newcommand{\prd}{Physical Review D}
\newcommand{\ssr}{Space Science Reviews}

% Symbols
\newcommand{\ydata}{\ensuremath{\boldsymbol{y}}}
\newcommand{\hyperparams}{\ensuremath{\boldsymbol{\phi}}}
\newcommand{\meanparams}{\ensuremath{\boldsymbol{\theta}}}
\newcommand{\dt}{\ensuremath{\tau}}
\newcommand{\amplitude}{\ensuremath{\alpha}}
\newcommand{\lengthscale}{\ensuremath{\lambda}}

\DeclareMathOperator*{\argmax}{arg\,max}

\newcommand{\project}[1]{\textsf{#1}}

% Document starts
\begin{document}

% Page header
\markboth{Kenworthy \& others}{HCI}

% Title
\title{High Contrast Imaging}

%Authors, affiliations address.
\author{Matthew Kenworthy,$^1$ Sebastiaan Haffert$^2$ and Emiel Por$^3$
  \affil{$^1$Leiden Observatory, Niels Bohrweg 2, Leiden 2300RA, The Netherlands; email: kenworthy@strw.leidenuniv.nl}
  \affil{$^2$Steward Observatory; email: haffert@astronomy.arizona.edu}
  \affil{$^3$STScI; email: por@stsci.edu}}

%Abstract
\begin{abstract}
High Contrast Imaging will enable the direct detection of photons from rocky terrestrial worlds.
\end{abstract}

%Keywords, etc.
\begin{keywords}
 Optics, coronagraphs, exoplanets, computational methods

\end{keywords}
\maketitle

%Table of Contents
\tableofcontents

\section{INTRODUCTION}
\label{sec:intro}

High contrast imaging, Lyot corongraph, 2000's development


In this review, we provide interactive notebooks that enable the reader to build up their intuitition on how coronagraphs work along with wavefront sensing both in the pupil and the focal planes, and how we will tackle the challenges in reaching the contrasts of $10^{-10}$ at angular separations of less than one arcsecond that are required.


\begin{armarginnote}[]
  \entry{HCI}{High Contrast Imaging}
  \entry{FP}{Focal Plane}
  \entry{PP}{Pupil Plane}
  \entry{FPWFS}{Focal Plane Wavefront Sensing}
  \entry{WFS}{Wavefront Sensor}
\end{armarginnote}

% text below from DFM
This manuscript was prepared using the \project{showyourwork} package\footnote{\url{https://show-your.work}} and the source code used to generate each figure is available in a public \project{GitHub} repository\footnote{\url{https://github.com/mkenworthy/ARAA_HCI}}.

To see the specific version of the \project{Jupyter} notebook, that was executed to generate each figure, click on the icon next to the figure caption.

\subsection{Brief history}


\begin{figure}[ht]
  \centering
  \script{plot_simple_psf.py}
  \includegraphics[width=1.0\linewidth]{figures/simple_psf.pdf}
  \caption{Figure run from a static Python script.}
  \label{fig:simplepsf}
\end{figure}


\lipsum[2-4]

\subsection{Motivating examples}
\label{sec:sim_examples}
\lipsum[2-4]


\section{CONCLUSIONS}
\label{sec:concl}
\lipsum[2-4]

\lipsum[2-4]


\subsection{Future perspectives}

\lipsum[2-4]

%Disclosure
\section*{DISCLOSURE STATEMENT}
The authors are not aware of any affiliations, memberships, funding, or financial holdings that
might be perceived as affecting the objectivity of this review.

% Acknowledgements
\section*{ACKNOWLEDGMENTS}
M.\ A.\ K.\ acknowledges useful conversations with
Phil Hinz.
% Eric Agol,
% Will Farr,
% Alex Gagliano,
% Tyler Gordon,
% and Maximiliano Isi.

% The authors would like to thank the community members who sent feedback on the public draft of this review:

To achieve the scientific results presented in this article we made use of the \emph{Python} programming language\footnote{Python Software Foundation, \url{https://www.python.org/}}, especially the \emph{SciPy} \citep{virtanen2020}, \emph{NumPy} \citep{numpy}, \emph{Matplotlib} \citep{Matplotlib}, \emph{emcee} \citep{foreman-mackey2013}, and \emph{astropy} \citep{astropy_1,astropy_2} packages.
%

This research has made use of NASA's Astrophysics Data System Bibliographic Services.

% References

\bibliographystyle{ar-style2}
\bibliography{bib}

% \section*{RELATED RESOURCES}

\end{document}

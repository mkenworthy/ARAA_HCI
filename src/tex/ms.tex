\documentclass[letterpaper]{ar-1col}
\usepackage{showyourwork}
\usepackage[letterpaper]{geometry}

\usepackage{natbib}
\usepackage{amsmath}
\usepackage{color}
\usepackage{hyperref}
\hypersetup{hidelinks}
\usepackage{nomencl}
\makenomenclature

\usepackage{graphbox}
\newcommand{\suz}[1]{\textcolor{magenta}{#1}}
\newcommand{\dan}[1]{\textcolor{green}{#1}}

\setcounter{secnumdepth}{4}
\usepackage{url}


\usepackage{lipsum}  


% Metadata Information
\jname{Annu. Rev. Astron. Astrophys.}
\jvol{AA}
\jyear{2024}
\doi{10.1146/TBD}

% autoref formatting
\def\sectionautorefname{Section}
\let\subsectionautorefname\sectionautorefname
\let\subsubsectionautorefname\sectionautorefname

% macros
\newcommand{\apjl}{Astrophysical Journal Letters}
\newcommand{\aj}{Astronomical Journal}
\newcommand{\apj}{Astrophysical Journal}
\newcommand{\apjs}{Astrophysical Journal Supplement}
\newcommand{\pasp}{Publications of the Astronomical Society of the Pacific}
\newcommand{\jgr}{Journal of Geophysical Research}
\newcommand{\aap}{Astronomy and Astrophysics}
\newcommand{\mnras}{Monthly Notices of the Royal Astronomical Society}
\newcommand{\actaa}{Acta Astronomica}
\newcommand{\nat}{Nature}
\newcommand{\prl}{Physical Review Letters}
\newcommand{\prd}{Physical Review D}
\newcommand{\ssr}{Space Science Reviews}
\newcommand{\araa}{Annual Review of Astronomy and Astrophysics}

% Symbols
\newcommand{\ydata}{\ensuremath{\boldsymbol{y}}}
\newcommand{\hyperparams}{\ensuremath{\boldsymbol{\phi}}}
\newcommand{\meanparams}{\ensuremath{\boldsymbol{\theta}}}
\newcommand{\dt}{\ensuremath{\tau}}
\newcommand{\amplitude}{\ensuremath{\alpha}}
\newcommand{\lengthscale}{\ensuremath{\lambda}}

\DeclareMathOperator*{\argmax}{arg\,max}

\newcommand{\project}[1]{\textsf{#1}}

% Comments:
\newcommand{\commentehp}[1]{\textcolor{red}{[EHP: #1]}}
\newcommand{\commentmak}[1]{\textcolor{red}{[MAK: #1]}}
\newcommand{\commentsyh}[1]{\textcolor{red}{[SYH: #1]}}

\newcommand{\notebooksuggestion}[1]{\textcolor{blue}{[Notebook: #1]}}

% Document starts
\begin{document}

% Page header
\markboth{Kenworthy \& others}{HCI}

% Title
\title{High Contrast Imaging}

%Authors, affiliations address.
\author{Matthew Kenworthy,$^1$ Sebastiaan Haffert$^2$ and Emiel Por$^3$
  \affil{$^1$Leiden Observatory, Niels Bohrweg 2, Leiden 2300RA, The Netherlands; email: kenworthy@strw.leidenuniv.nl}
  \affil{$^2$Steward Observatory; email: haffert@astronomy.arizona.edu}
  \affil{$^3$STScI; email: epor@stsci.edu}}

%Abstract
\begin{abstract}
High Contrast Imaging will enable the direct detection of photons from rocky terrestrial worlds.
\end{abstract}

%Keywords, etc.
\begin{keywords}
 Optics, coronagraphs, exoplanets, computational methods

\end{keywords}
\maketitle

%Table of Contents
\tableofcontents

\section{INTRODUCTION}
\label{sec:intro}

% High contrast imaging, Lyot corongraph, 2000's development


In this review, we provide interactive notebooks that enable the reader to build up their intuition on how coronagraphs work along with wavefront sensing both in the pupil and the focal planes, and how we will tackle the challenges in reaching the contrasts of $10^{-10}$ at angular separations of less than one arcsecond that are required.


\begin{armarginnote}[]
  \entry{HCI}{High Contrast Imaging}
  \entry{FP}{Focal Plane}
  \entry{PP}{Pupil Plane}
  \entry{FPWFS}{Focal Plane Wavefront Sensing}
  \entry{WFS}{Wavefront Sensor}
\end{armarginnote}

% text below from DFM
This manuscript was prepared using the \project{showyourwork} package\footnote{\url{https://show-your.work}} and the source code used to generate each figure is available in a public \project{GitHub} repository\footnote{\url{https://github.com/mkenworthy/ARAA_HCI}}.

To see the specific version of the \project{Jupyter} notebook, that was executed to generate each figure, click on the icon next to the figure caption.

\section{Science Goals}

Initially developed to image the Sun's corona without the need for a Solar eclipse, one of the most significant science drivers for the latest coronagraphs is in the detection and characterisation of circumstellar material and planets around nearby stars.
%
Planetary systems that are closer to the Sun have two benefits: given two identical planetary systems, one that is twice as close as the other will have twice the angular separation between the star and planet, and from the inverse square law, four times more flux is received from the planet.

For direct imaging, therefore, the closest stars to the Sun are the ones that are studied for the presence of directly imaged exoplanets.
%
In a volume limited (20 parsecs) sample of stars around the Sun \citep{2023arXiv231203639K}, there are XXXX solar (G2) type stars, and YYYY M-dwarf stars.
%
DI young Jupiters look are self-luminous in the NIR, but for terrestrial planets, looking for light reflected from the atmosphere/surface, so subject to inverse square law for planet separation from star.

Two regimes: solar analogues at optical wavelengths lead to space based missions, ground based larger aperture telescopes study M dwarf stars that are on average closer to the Sun, and look in the NIR.

Mention HABCAT for a catalogue of solar like stars.

Bioactive molecules see reviews of \citet{2016AsBio..16..465S,2017ARAA..55..433K,2018AsBio..18..663S}.


\section{Brief history}


\begin{figure}[ht]
  \centering
  \script{plot_simple_psf.py}
  \includegraphics[width=1.0\linewidth]{figures/simple_psf.pdf}
  \caption{Figure run from a static Python script.}
  \label{fig:simplepsf}
\end{figure}



% \begin{figure}[ht]
%   \centering
%   \script{interactive_fresnel.ipynb}
%   \includegraphics[width=1.0\linewidth]{figures/test.pdf}
%   \caption{Figure run from a Jupyter notebook.}
%   \label{fig:fresnel}
% \end{figure}



The scientific goal is to image an exoplanet (a planet orbiting a star other than our Sun) and to characterise its orbit and atmosphere.
%
For an Earth analogue with similar radius, albedo and effective temperature orbiting around a solar-type star 10 parsecs away, the typical amount of reflected light in the optical is $10^{-10}$ of the central star at a separation of 0.1 arcseconds at maximum elongation.
%
The technical challenge is in distinguishing the light of the parent star from the light of the planet.

\subsubsection{From Maxwell's Equations to Wavefronts}
The vast majority of energy from astrophysical objects arrives at our telescopes in the form of electromagnetic radiation.
%
The electromagnetic waves propagate through space by exchanges between the electric and magnetic fields because a changing electric field induces a magnetic field and vice-versa.
%
This time-dependent interaction is described by Maxwell's equations,

\begin{equation}
\begin{aligned}
\frac{\partial\mathcal{D}}{\partial t} \quad & = \quad \nabla\times\mathcal{H},   & \quad \text{(Faraday's law)} \\[5pt]
\frac{\partial\mathcal{B}}{\partial t} \quad & = \quad -\nabla\times\mathcal{E},  & \quad \text{(Ampere's Law)}   \\[5pt]
\nabla\cdot\mathcal{B}                 \quad & = \quad 0,                         & \quad \text{(Gauss's law)}   \\[5pt]
\nabla\cdot\mathcal{D}                 \quad & = \quad 0.                         & \quad \text{(Coulomb's law)}
\end{aligned}
\end{equation}
This form of Maxwell's equations is in the material form without any charge and current sources. The material form is used to describe the propagation of electromagnetic fields inside matter. This set of equations is completed by describing the particular matter of the medium with the constitutive relations, $D=\epsilon E$ and $B=\mu H$. Here $\epsilon$ and $\mu$ are the permittivity and magnetic permeability, respectively. The wave equation for electromagnetic waves can be derived by taking the curl of Ampere's law. This gives us the classic wave equation if we assume that the EM wave propagates in isotropic and homogeneous materials,
\begin{equation}
\label{eq:wave_eq}
\begin{aligned}
%\nabla\times\frac{\partial\mathcal{B}}{\partial t} \quad  =& \quad -\nabla\times\nabla\times\mathcal{E} \\
%\frac{\partial\nabla\times\mathcal{B}}{\partial t} \quad  =& \quad -\nabla^2\mathcal{E} \\
%\frac{\partial\nabla\times\mathcal{\mu H}}{\partial t} \quad  =& \quad -\nabla^2\mathcal{E} \\
%\mu\frac{\partial^2 \mathcal{D}}{\partial t^2} \quad  =& \quad -\nabla^2\mathcal{E} \\
\mu \epsilon \frac{\partial^2 \mathcal{E}}{\partial t^2} - \nabla^2\mathcal{E} = 0 .
\end{aligned}
\end{equation}
Let's now consider a purely monochromatic EM wave with frequency $\omega$. The wave function is then $\mathcal{E}=\psi(r) e^{iwt}$ with $\psi(r)$ describing the spatial distribution. Substituting this relation into Equation \ref{eq:wave_eq},
\begin{equation}
\nabla^2\mathcal{E} +\mu \epsilon \omega^2 \mathcal{E} = 0.
\end{equation}
The usual definition of the permittivity and permeability are $\epsilon=\epsilon_r \epsilon_0$ and $\mu=\mu_r \mu_0$ with $\epsilon_r$ and $\mu_r$ the relative permittivity and permeability compared to that of vacuum, $\epsilon_0$ and $\mu_0$. In optics most glasses and materials are defined by their refractive index $n$ which is related to permittivity as $\epsilon_r = n^2$ and many of these are also non-magnetic which means that $\mu_r=1$. Substituting this will give us the classic Helmholtz equation
\begin{equation}
\nabla^2\mathcal{E} + n^2k^2 \mathcal{E} = 0,
\end{equation}
We made use of the fact that the speed of light is $c = \frac{1}{\sqrt{\epsilon_0\mu_0}}$ and that $k = c\omega$ is the wave number. The propagation through an optical system has a preferential direction that is usually defined along the z-axis. The z-axis evolution can be derived by separating the spatial components into the z component and the perpendicular components (x,y),
\begin{equation}
\frac{\partial^2\mathcal{E}}{\partial z^2} = -\nabla_{\perp}^2\mathcal{E}-n^2k^2 \mathcal{E},
\end{equation}
This differential equation can be solved by assuming a plane wave expansion $\mathcal{E}(x,y,z)=e^{i(k_x x + k_y y)}f(z)$ which results in
\begin{equation}
\frac{\partial^2\mathcal{E}}{\partial z^2} = -(n^2k^2 - k_{\perp}^2)\mathcal{E}.
\end{equation}
The solution to this equation is the so called Angular Spectrum Propagator that relates the electric field at any one plane to the electric field at any other,
\begin{equation}
\label{eq:angular_spectrum}
\mathcal{E}(x, y, z') = \mathcal{F}_{x,y}^{-1}\{e^{-ik_z(z'-z)}\mathcal{F}_{x,y}\{\mathcal{E}(x,y,z)\}\}.
\end{equation}

Here the z component of the wave vector is defined as $k_z=\sqrt{n^2k^2 - k_{\perp}^2}$ and $\mathcal{F}_{x,y}^{(-1)}$ is defined as the (inverse) Fourier transform over the x and y coordinates. While Equation \ref{eq:angular_spectrum} describes the full propagation from one plane to another, it is quite unwieldy to use and does not provide much physical insight. For many optical systems it is sufficient to analyze the paraxial performance. The paraxial approximation assumes that the plane waves make small angles with respect to the z-axis which means that the x and y wave vector components are $k_x,k_y<<1$. In this regime the propagation factor $k_z=\sqrt{n^2k^2 - k_{\perp}^2}\approx nk - \frac{k_{\perp}^2}{2nk}$
\begin{equation}
\label{eq:angular_spectrum2}
\mathcal{E}(x, y, z') = e^{-ink(z'-z)} \mathcal{F}_{x,y}^{-1}\{\mathcal{F}_{x,y}\{\mathcal{E}(x,y,z)\}\}.
\end{equation}
%$$E(\mathbf{r},t)=E_0(\mathbf{r})e^{i(\mathbf{k}.\mathbf{r}-\omega t)}$$

%Maxwell's equations describe the time-dependent interaction between static and moving electric charges, electric fields and magnetic fields.


%When a star is imaged with a large ground based telescope the detector does not see a point source with all flux concentrated into one pixel, but instead the stellar flux is distributed across a finite area.%actually an infinite area because band-limited signals have an infinite support. There is no end to the Airy pattern.

%
%This is due to the wave-like nature of electromagnetic radiation.


% ENERGY PROPAGATION
%Energy is stored in both the electric and magnetic fields.
% WAVE SOLUTIONS
%One solution to these equations are second-order differential wave equations, which describe a periodic change in the electric and magnetic field strength throughout a given volume.
%
%In the absence of any free electrical charges, these show that, energy is propagated in a direction along their direction of motion. BLAH NOT TRUE FOR CALCITE...
% ENERGY IN E FIELDS
%The majority of energy is stored in the electric field, and since the magnetic field strength follows the electric field strength we only talk about the time and space evolution of the electric field.

%The time-varying electric field strength of an electromagnetic plane wave at time $t$ is described by:

%$$E(\mathbf{r},t)=E_0(\mathbf{r})e^{i(\mathbf{k}.\mathbf{r}-\omega t)}$$
%where $\mathbf{r}$ is a point in three-dimensional space, $\mathbf{k}$ is a unit vector pointing in the direction of the wave propagation, and $\omega$ is the angular frequency of the wave.
%
%The angular frequency and wavelength of the wave is related by $\omega = 2\pi c/\lambda$ where $c$ is the propagation speed in free space.
%
% PHASE only applies to a periodic wave!

%Waves are periodic functions, so we introduce the concept of phase: the distance (in space or time) from a reference point in the amplitude of wave, usually defined so that a phase of zero is the place/time where the amplitude of the field is the largest. UGH NOT QUITE RIGHT. 

%A wavefront is a surface with constant phase.
%
%For an electromagnetic wave propagating through free space, it is a flat two dimensional plane.
%
%This plane moves in the direction of propagation $k$.

%Optical elements bring separate spatial regions of the wave to the same spatial point.
%
%Since the wave is coherent, the resultant electric field is a linear addition of the electric field propagated along the optical path length difference.
% Maxwell's equations are linear and so superposition of different k vectors and different omegas are possible
%
% The wave propagation has k = E x B, so all three quantities are perpendicular to each other.

%There are two orthogonal solutions, corresponding to two independent polarizations for electromagnetic waves.

%Ultimately, if we know the electric field distribution at the telescope aperture then we can propagate the resultant electromagnetic waves through our telescope and instrument to subsequent focal and pupil planes.

The coherent superposition of electric fields from different parts of the telescope aperture and the associated phase offset gives rise to \textbf{interference}, where the sum of all these electric amplitudes can give very different results from an incoherent addition of the fluxes.

\section{Definition of quantities}

\nomenclature{$i$}{The imaginary unit}

\nomenclature{$c$}{Speed of light in a vacuum}
\nomenclature{$\vec{E}$}{Electric field}
\nomenclature{$\vec{H}$}{Magnetic field}
\nomenclature{$\mathcal{D}$}{Displacement field}
\nomenclature{$\mathcal{B}$}{Magnetizing field}
\nomenclature{$\epsilon$}{Electric permittivity}
\nomenclature{$\epsilon_0$}{Electric permittivity of vacuum}
\nomenclature{$\mu$}{Magnetic permittivity}
\nomenclature{$\mu_0$}{Magnetic permittivity of vacuum}
\nomenclature{$\omega$}{Wavenumber}
\nomenclature{$\vec{k}$}{Wavevector}

\nomenclature{$\lambda$}{Wavelength}
\nomenclature{$\lambda_0$}{Center wavelength in the bandpass}
%\nomenclature{$\delta\lambda$}{d}
\nomenclature{$\Delta\lambda$}{The width of the bandpass}

\nomenclature{$D$}{Diameter of the pupil}

\nomenclature{$n$}{(Complex) refractive index of a macroscopic material}
\nomenclature{$\mathcal{F}_{x,y}[\cdot]$}{Fourier transform operator}
\nomenclature{$\mathcal{F}^{-1}_{x,y}[\cdot]$}{Inverse Fourier transform operator}
\nomenclature{$C_\lambda[\cdot]$}{A general coronagraph propagation operator}

\nomenclature{$\Psi_\lambda[\vec{k}]$}{Coronagraphic image}
\nomenclature{$\phi$}{The phase of the electric field}
\nomenclature{$\alpha$}{The amplitude of the electric field}
\nomenclature{$\Pi$}{The telescope pupil function.}

\printnomenclature

\section{Diffraction of simple apertures}
\notebooksuggestion{different apertures and geometries}

A square aperture with a side length of $D$ illuminated by light of wavelength $\lambda$ will produce the illumination seen in Figure XXXX.
%
The focal plane is divided by a rectilinear dark grid
%
The angular distance from the optical axis to the first dark minimum is $\lambda/D$ in units of radians.

For a circular unobstructed telescope aperture, this pattern is referred to as the Airy function.

\section{The Lyot Coronagraph}

\notebooksuggestion{lyot fully adjustable}
Stellar coronagraphs have now been in use for several decades. However, the first coronagraph was developed by Bernard Lyot to observe the corona of the Sun in the 1930s. It took over half a century before astronomers applied coronagraphs to image the faint cirumstellar environment by blocking starlight. 

The first coronagraph to successfully image a debris disk was a Lyot coronagraph built by \citet{Vilas87} and it imaged the edge-on circumstellar disk around Beta Pictoris in 1984 \citep{Smith84}.
%
The optical layout of the Lyot coronagraph can be generalised in Figure XXXXX, with the letters A-F representing the images present at that location in the coronagraph light path.
%
The telescope pupil (A) is reimaged into a focal plane of the sky (B) where an opaque mask that has high absorptivity and low reflectivity blocks the light from any on-axis source (C).
%
Optics then form an image of the resultant pupil (D) to an intermediate pupil plane, where a second opaque mask - the Lyot stop - is located.
%
This Lyot stop blocks the diffracted light of the star - now forming a ring of light along the outer edge of the telescope pupil (E).
%
A second optical system then reimages the resultant focal plane of the instrument onto a detector (F).
%
%When an object is placed behind the focal plane mask, the light from that object is blocked and does not go any further.
%
Any circumstellar objects outside the radius of the focal plane mask then pass through unimpeded through the coronagraph and are subsequently reimaged in the detector focal plane at F.
%
The light rays pass through the coronagraph optics to form an image at F with only minor modification: the reimaged pupil at D is superficially very similar to the telescope pupil A.

For an on-axis source, the removal of the Airy core plus attendant diffraction rings significantly modifies the wavefront passing through the coronagraph, resulting in a flux redistribution at D where the flux is concentrated in a ring whose peak brightness lies along the perimeter of the reimaged telescope pupil, extending both beyond the radius of the pupil and into the centre of the pupil.

The purpose of the Lyot stop is to remove as much of this ring of light as possible, whilst maximising the throughput of the pupil image D for off-axis sources.
%
Adjusting the diameter of the focal plane mask changes the full width half maximum of the ring of light at D, which requires a smaller Lyot stop to block - but then the throughput of the pupil for off-axis sources then decreases.
%
Decreasing the Lyot stop aperture has a second impact in that the reduced pupil diameter increases the Full Width Half Maximum (FWHM) of the images in the final focal plane F, spreading the flux from the off-axis sources over a larger area in the detector and degrading the angular resolution of the telescope and instrument.
%
The optimal diameters of the Focal Plane Mask (FPM) and Lyot stop aperture are then driven by the science requirements - how close to the central star (measured in diffraction widths at the lower spatial resolution) should the coronagraph be able to transmit light from off-axis objects in the field of view.



\section{Focal plane phase mask coronagraphs}
The Lyot coronagraph was designed at the time when it was still difficult to precisely manipulated the phase of light with optics. With the advent of more advanced manufacturing capabilities, very precise phase control became possible. This was a significant boost to the design toolbox for coronagraphs. The major downside of the classic Lyot coronagraph is that only the light that falls on the opaque focal plane masks gets blocked. Any light that is off-axis will pass through the system. This holds not only for light that comes from off-axis sources but also for aberrations that causes light to end up outside of the mask. A bigger mask will be able to block a larger fraction of the light and therefore achieve a deeper contrast. However, if the mask is made larger the inner-working angle also becomes bigger and that means fewer planets will be accessible. A smaller inner-working angle is crucial both for the extremely large ground-based telescopes and the space-based telescopes. 

Focal plane phase masks offer a solution. Phase masks don't enhance the contrast by blocking light but by phase shifting a part of the PSF (usually the core of the PSF). The phase shifted part then causes destructive interference at the Lyot plane. The Lyot stop blocks the areas where the light does not destructively interferes. The 50\% encircled energy radius is on the order of $\sim \lambda/D$ for almost all aperture shapes. This means that it is possible to achieve perfect destructive interference with a mask that has a size on the order of $\lambda/D$. This is the central idea that was used to design the Roddier and Roddier (RR) phase mask \cite{roddier1997stellar}. The RR mask covers the core of the Airy pattern that contains 50\% of the encircled energy and phase shifts the core by $\pi$ to cause destructive interference. The RR mask works well for monochromatic light. However, over broad spectral bandwidths it degrades in contrast. Diffraction causes wavelength scaling of the PSF that either makes the focal plane mask too big or too small compared to the PSF. Masks made out of multiple concentric rings, such as the dual-zone phase mask \cite{soummer2003achromatic}, were proposed to increase the spectral bandwidth.

Further achromatization for larger spectral bandwidths is possible by designing inherent achromatic phase masks. Inherent achromatic masks are scale invariant, which means that they always look the same regardless of the size of the PSF. One of the most used phase mask coronagraphs is the vortex coronagraph (VC). The VC uses a phase mask with a vortex pattern where the phase changes with the azimutal angle: $\phi=q \cdot \theta$ with $q$ the charge, i.e. the number of times the phase wraps around, of the vortex and $\theta$ the azimuth angle. Two examples of scale invariant phase masks are shown in Figure \ref{fig:scale_invariant}.

\begin{figure}[ht]
  \centering
  \script{plot_scale_invariant_masks.py}
  \includegraphics[width=0.8\linewidth]{figures/scale_invariant_masks.pdf}
  \caption{The focal plane phase functions of two scale invariant coronagraphs. The left figure shows the four quadrant phase mask and the right figure shows the optical vortex phase mask.}
  \label{fig:scale_invariant}
\end{figure}

Both phase masks that are shown in Figure \ref{fig:scale_invariant} completely null out all on-axis light from a clear aperture. This property is why both the FQPM and the OVC have been extensively studied over the past several decades. The inner-working angle of both coronagraph approaches 1 $\lambda/D$ which makes coronagraphy possible at the diffraction limit! The FQPM has been implemented on SPHERE 
 
%All ground-based telescopes need to image through Earth's atmosphere. The turbulent nature of the atmosphere causes aberrations that degrade the imaging qualities of the telescope. In such conditions, it is very difficult to optimally reject starlight because what electric field actually needs to be rejected? Is it the electric field that is measured now, or the differently aberrated electric field 20 milliseconds from now? The constantly changing electric field of the incoming light made it difficult to create more advanced coronagraphs. The classic Lyot coronagraph design was not improved upon until the advent of large aperture telescopes with adaptive optics systems that had a high enough performance to deliver diffraction-limited images.
%
%These telescopes made the imaging of low mass substellar companions (brown dwarfs and young self-luminous planets) feasible for the first time.

%
%After the discovery of Gliese 229B (XXX CHECK WAS THIS A LYOT?) with the Palomar 5m PALM 5000 system, a series of coronagraphic designs appeared where researchers explored the different possibilities of placing complex amplitude apodizers at both the focal plane and pupil plane locations.
%
%A wide range of coronagraph designs were explored, including XXXXX focal plane, pupil plane, telephone list of all different coronagraphs.
%
%MENTION interferometer concepts.

%All stars have small but finite angular diameter on the sky, typically $\lambda/100$ or less but $\lambda/10$ is possible for the closest stars. \commentehp{In the visible, median stars for HWO are going to be $\lambda/10$, with some a little larger.} \commentsyh{And for GMT and ELT some are even going to be bigger. Prox Cen is 1 mas which is 1/3 to 1/6 of lambda/D for ELT and GMT.}
%

%\lipsum[2-4]

\section{Parameters to optimize}
\lipsum[2-4]

\section{Two approaches: ground and ELT for M dwarf}

\notebooksuggestion{play with different contrasts and IWA to see yields}

\lipsum[2-4]

\section{Space coronagraphs and HWO for G type stars}
\lipsum[2-4]

\section{Polarization effects} 
\notebooksuggestion{Polarimetric beam shift using Fresnel equation - Change angle of incidence and show differential beam shift.}
\lipsum[2-4]

\section{PIAA}
\notebooksuggestion{How do you make a PIAA surface? And show improved performance w.r.t. standard lyot coronagraph.}
\lipsum[2-4]

\section{Coronagraphic wavefront sensing}
\lipsum[2-4]

\section{Focal plane wavefront sensing}

\notebooksuggestion{Simple dark hole generation with predefined shapes - can use the new autodiff approach.}

\lipsum[2-4]

\section{rejected light wavefront sensing} 
\lipsum[2-4]

\section{PAPLC}
\lipsum[2-4]

\section{Segmented mirrors}
\lipsum[2-4]

\section{Integrated optics}
\lipsum[2-4]

\section{Astrophotonics and QOD}
\lipsum[2-4]

\section{Quantum optimal detection}
\lipsum[2-4]

\section{Post-processing (its no longer a photon)}

\notebooksuggestion{KLIP with removing varying number of modes and annular rings.}

\lipsum[2-4]

%Disclosure
\section*{DISCLOSURE STATEMENT}
The authors are not aware of any affiliations, memberships, funding, or financial holdings that
might be perceived as affecting the objectivity of this review.

% Acknowledgements
\section*{ACKNOWLEDGMENTS}
M.\ A.\ K.\ acknowledges useful conversations with
Phil Hinz.
% Eric Agol,
% Will Farr,
% Alex Gagliano,
% Tyler Gordon,
% and Maximiliano Isi.

% The authors would like to thank the community members who sent feedback on the public draft of this review:

To achieve the scientific results presented in this article we made use of the \emph{Python} programming language\footnote{Python Software Foundation, \url{https://www.python.org/}}, especially the \emph{SciPy} \citep{virtanen2020}, \emph{NumPy} \citep{numpy}, \emph{Matplotlib} \citep{Matplotlib}, \emph{emcee} \citep{foreman-mackey2013}, and \emph{astropy} \citep{astropy_1,astropy_2} packages.
%

This research has made use of NASA's Astrophysics Data System Bibliographic Services.

% References

\bibliographystyle{ar-style2}
\bibliography{bib}

% \section*{RELATED RESOURCES}

\end{document}
